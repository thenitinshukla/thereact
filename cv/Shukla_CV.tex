%%% LaTeX Template: Designer's CV
%%%
%%% Source: http://www.howtotex.com/
%%% Feel free to distribute this template, but please keep the referal to HowToTeX.com.
%%% Date: March 2012


%%%%%%%%%%%%%%%%%%%%%%%%%%%%%%%%%%%%%
% Document properties and packages
%%%%%%%%%%%%%%%%%%%%%%%%%%%%%%%%%%%%%
\documentclass[a4paper,10pt,final]{memoir}

% misc
\renewcommand{\familydefault}{bch}	% font
\pagestyle{empty}					% no pagenumbering
\setlength{\parindent}{0pt}			% no paragraph indentation

% required packages (add your own)
\usepackage{flowfram}										% column layout
\usepackage[top=2.5cm,left=0.5cm,right=0.5cm,bottom=2.5cm]{geometry}% margins
\usepackage{graphicx}										% figures
\usepackage{url}											% URLs
\usepackage[usenames,dvipsnames]{xcolor}					% color
\usepackage{multicol}										% columns env.
	\setlength{\multicolsep}{0pt}
\usepackage{paralist}										% compact lists
\usepackage{tikz}
\usepackage{eurosym}

%%%%%%%%%%%%%%%%%%%%%%%%%%%%%%%%%%%%%
% Create column layout
%%%%%%%%%%%%%%%%%%%%%%%%%%%%%%%%%%%%%
% define length commands
\setlength{\vcolumnsep}{\baselineskip}
\setlength{\columnsep}{\vcolumnsep}

% frame setup (flowfram package)
% left frame
\newflowframe{0.24\textwidth}{\textheight}{0pt}{0pt}[left]
	\newlength{\LeftMainSep}
	\setlength{\LeftMainSep}{0.24\textwidth}
	\addtolength{\LeftMainSep}{1\columnsep}
 
% small static frame for the vertical line
\newstaticframe{1.5pt}{\textheight}{\LeftMainSep}{0pt}
 
% content of the static frame
\begin{staticcontents}{1}
\hfill
\tikz{%
	\draw[loosely dotted,color=RoyalBlue,line width=1.5pt,yshift=0]
	(0,0) -- (0,\textheight);}%
\hfill\mbox{}
\end{staticcontents}
 
% right frame
\addtolength{\LeftMainSep}{1.5pt}
\addtolength{\LeftMainSep}{1\columnsep}
\newflowframe{0.7\textwidth}{\textheight}{\LeftMainSep}{0pt}[main01]


%%%%%%%%%%%%%%%%%%%%%%%%%%%%%%%%%%%%%
% define macros (for convience)
%%%%%%%%%%%%%%%%%%%%%%%%%%%%%%%%%%%%%
\newcommand{\Sep}{\vspace{0.8cm}}
\newcommand{\SmallSep}{\vspace{0.3cm}}

\newenvironment{AboutMe}
	{\ignorespaces\textbf{\color{RoyalBlue} About me}}
	{\Sep\ignorespacesafterend}
	
\newcommand{\CVSection}[1]
	{\Large\textbf{#1}\par
	\SmallSep\normalsize\normalfont}

\newcommand{\CVItem}[1]
	{\textbf{\color{RoyalBlue} #1}}


%%%%%%%%%%%%%%%%%%%%%%%%%%%%%%%%%%%%%
% Begin document
%%%%%%%%%%%%%%%%%%%%%%%%%%%%%%%%%%%%%
\begin{document}

% Left frame
%%%%%%%%%%%%%%%%%%%%
\begin{figure}
	\hfill
	\includegraphics[width=0.6\columnwidth]{me}
	\vspace{-7cm}
\end{figure}

\begin{flushright}\scriptsize
	Nitin Shukla\\
	SuperComputing Applications\\
	and Innovation Department\\
	CINECA\\
	via Magnanelli 6/3 \\
	40033 Casalecchio di Reno\\
	Italy\\
	email: nshukla@tecnico.ulisboa.pt\\
	mobile: +393516706185\\
	skype: nitshukla
\end{flushright}\normalsize
\framebreak


% Right frame
%%%%%%%%%%%%%%%%%%%%
\Huge\bfseries {\color{RoyalBlue} Nitin Shukla} \\
%\Large\bfseries  Graphics designer \\

\normalsize\normalfont

% About me
\begin{AboutMe}
I am an enthusiastic and motivated researcher, willing to acquire new skills and expertise to expand my horizons. I am passionate about data science and High Performance Computing. I have proven experience in working independently and in teams. I am a quick learner and an ambitious result-driven hard worker.
%I routinely teach parallel computing courses such as OpenMP and OpenACC. I have experience in mentoring parallelising serial codes to both the CPU and the GPUs.
\end{AboutMe}

\vspace{-0.5cm}
% Skills
\CVSection{Demonstrated Skills}

\CVItem{Problem solving and strategic thinking}
\begin{compactitem}[\color{RoyalBlue}$\circ$]
\item Development and employment of mathematical modelling and computational techniques for investigating physics problems
\item Analysis of complex data structures
\item Use of my experience and knowledge to work around obstacles
\end{compactitem}
\Sep

\CVItem{Organizational skills}
\begin{compactitem}[\color{RoyalBlue}$\circ$]
\item Ability to plan and prioritise tasks with the aim of meeting deadlines
\item Logistic abilities proven by the organisation of postgraduate courses about parallel programming
\end{compactitem}
\Sep

\CVItem{Team work and leadership skills}
\begin{compactitem}[\color{RoyalBlue}$\circ$]
\item Ability to carry out interdisciplinary projects involving several researchers with different expertise, as demonstrated by my publication record
\item Ability to effectively guide teams, as proven by my mentoring roles in NVIDIA hackathon
\item Ability to work in a multi-cultural environment, as evidenced by the fact that my work activities have been conducted in five different countries
\end{compactitem}
\Sep

\CVItem{Communication and presentation skills}
\begin{compactitem}[\color{RoyalBlue}$\circ$]
\item Highly trained in delivering presentations, as demonstrated by my participation to international conferences
\item Ability to explain difficult concepts to non-specialists, as also shown by the layman articles that I write on medium.com
\item Ability to teach courses on parallel programming for researchers
\end{compactitem}
\Sep

\CVItem{Writing skills}
\begin{compactitem}[\color{RoyalBlue}$\circ$]
\item Concise and clear writing style as shown by my 29 scientific publications in peer-reviewed international journals
\item Trained in proposal and grant writing with proven track record of success in attracting monetary and computational fundings for my research 
\end{compactitem}
\Sep

\CVItem{IT Skills}
\begin{compactitem}[\color{RoyalBlue}$\circ$]
\item Deep knowledge of UNIX/LINUX systems and shell scripting
\item Routine employment of Python, IDL, Matlab and Mathematica
\item Very competent on JuliaLang, Fortran, C and C++ programming languages
\item Able to develop parallel codes using MPI, OpenMP, OpenACC and CUDA
\item Experience in the use of version control softwares as git and svn
\item Excellent knowledge of the Office suite and its equivalent for Mac and LINUX  
\item Routine employment of \LaTeX
\item Proficient in HTML and CSS
\item Basic knowledge of SQL
\item Basic usage of Adobe Illustrator
\end{compactitem}

\clearpage
\framebreak
\framebreak

\CVSection{Education}

I have two independent PhD degrees.
\SmallSep

\CVItem{2013-2019, PhD II in computational plasma physics at Instituto Superior T\'ecnico (Lisbon, Portugal)}\\
My studies were based on massively parallel numerical simulations. I gained experience in deploying numerical codes and supporting libraries on a variety of platforms. I wrote scripts in Python, IDL and Matlab to analyse the data obtained from my simulations.\\
Passed with Distinction.
\SmallSep

\CVItem{2010-2012, PhD I in theoretical plasma physics at Ume\r{a} Universitet (Ume\r{a}, Sweden)}\\
I developed theoretical models to describe non-linear plasma dynamics. I wrote scripts in Mathematica to solve complex equations.
\SmallSep

\CVItem{2008-2010, Master Degree in Physics Engineering at Instituto Superior T\'ecnico (Lisbon, Portugal)}\\
Passed First class.
\Sep

\CVSection{Experience}
\CVItem{2020-today, High Performance Computing analyst at CINECA (Casalecchio di Reno, Italy)}\\
User support for the Eurofusion project, PI of the project try21 to port the code ECsim to GPU using OpenACC, co-developer of the CUDA version of the XShell code (H2020 ChEESE project), technical referee for PRACE and ISCRA projects, organiser and convener of courses on JuliaLang, C++ and OpenMP, host of HPC-Europa3 projects.
\SmallSep

\CVItem{2019-2020, Postdoctoral researcher at Instituto Superior T\'ecnico (Lisbon, Portugal)}\\
Massively parallel numerical simulations based on the Particle-In-Cell technique.
\SmallSep

\CVItem{2016, Websummit (Lisbon, Portugal)}\\
Representative of IB hubs. 
\SmallSep

\CVItem{2006-2007, Visiting researcher at Ruhr-Universit\"{a}t (Bochum, Germany)}\\
Development of mathematical models to describe waves and instabilities in plasmas. 
\Sep

\CVSection{Prizes and awards}
\begin{compactitem}[\color{RoyalBlue}$\circ$]
\item Best scientific visualisation ARCHER2 HPC Image and Video Competition, \pounds 250 (2020)
\item HPC-Europa3 project (2019), 500,000 CPUhours on ARCHER + \pounds 2,000 to visit Lancaster University (Lancaster, UK)
\item HPC-Europa3 project (2018), 1,250,000 CPUhours on Marconi + \euro 1,500 to visit CINECA (Casalecchio di Reno, Italy)
\item Best scientific visualisation GoLP Image and Video Competition, \euro 500 (2015)
\end{compactitem}
\Sep

\CVSection{Languages}
\begin{compactitem}[\color{RoyalBlue}$\circ$]
\item Hindi (mother tongue)
\item English (C1)
\item Portuguese (A2)
\item Italian (A2)
\end{compactitem}
\Sep

\CVSection{Publications}
For the full list of publications, please refer to my Google Scholar profile:\\ 
\url{https://scholar.google.com/citations?user=Fs_mg34AAAAJ&hl=pt-PT}.\\ 

%\Sep

%% References
%\CVSection{References}
%References upon request.

%%%%%%%%%%%%%%%%%%%%%%%%%%%%%%%%%%%%%
% End document
%%%%%%%%%%%%%%%%%%%%%%%%%%%%%%%%%%%%%
\end{document}